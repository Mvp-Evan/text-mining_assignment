% Preamble
\documentclass[conference]{IEEEtran}
\IEEEoverridecommandlockouts

% Packages
\usepackage{cite}
\usepackage{amsmath,amssymb,amsfonts}
\usepackage{algorithmic}
\usepackage{graphicx}
\usepackage{textcomp}
\usepackage{xcolor}
\usepackage{hyperref}

\def\BibTeX{{
        \rm B\kern-.05em{\sc i\kern-.025em b}\kern-.08em T\kern-.1667em\lower.7ex\hbox{E}\kern-.125emX
    }
}

\title{Text-Mining Short Paper}
\author{
    \IEEEauthorblockN{
        Qixuan Yang
    }
    \IEEEauthorblockA{
        \textit{
            Computer Science
        } \\
        \textit{
            University of Manchester
        }\\
        Manchester, the United Kingdom \\
        qixuan.yang@postgrad.manchester.ac.uk
    }
    \and
    \IEEEauthorblockN{
        Hong Wang
    }
    \IEEEauthorblockA{
        \textit{
            Computer Science
        } \\
        \textit{
            University of Manchester
        }\\
        Manchester, the United Kingdom \\
        hong.wang-5@student.manchester.ac.uk
    }
    \and
    \IEEEauthorblockN{
        Matei-Alexandru Costin
    }
    \IEEEauthorblockA{
        \textit{
            Computer Science
        } \\
        \textit{
            University of Manchester
        }\\
        Manchester, the United Kingdom \\
        matei-alexandru.costin@student.manchester.ac.uk
    }
    \and
    \IEEEauthorblockN{
        Jianxin Yu
    }
    \IEEEauthorblockA{
        \textit{
            Computer Science
        } \\
        \textit{
            University of Manchester
        }\\
        Manchester, the United Kingdom \\
        jianxin.yu@postgrad.manchester.ac.uk
    }
}

%%% BEGIN DOCUMENT
\begin{document}
\maketitle

\begin{abstract}
This document is a model and instructions for \LaTeX.
This and the IEEEtran.cls file define the components of your paper [title, text, heads, etc.]. *CRITICAL: Do Not Use Symbols, Special Characters, Footnotes, 
or Math in Paper Title or Abstract.
\end{abstract}

\begin{IEEEkeywords}
component, formatting, style, styling, insert
\end{IEEEkeywords}

% Requirements:
% The paper content should be at most four pages long, plus any number of pages for references and any appendices.

\section{Introduction}
% (1) provide an introduction to the RE task and describe your chosen RE dataset; 

DocRed dataset\cite{yao2019docred}

\section{Literature Review}
% (2) review previously reported work that is related to your approaches; 

DocRed dataset with LSTM\cite{yao2019docred}

CNN\cite{yao2019docred}

BiLSTM\cite{yao2019docred}

SAIS-BERT\cite{xiao2021sais}

GAIN-BERT\cite{zeng2020double}

Two-steps BERT\cite{wang2019fine}

DREEAM\cite{ma2023dreeam}

SSAN-RoBERTa-large+Adaptation\cite{xu2021entity}

KD-Rb-l\cite{tan2022document}

\section{Approaches}
% (3) describe the two RE approaches that you developed, justifying why they were chosen and detailing how they were implemented; 

Two RE approaches: LSTM\cite{yao2019docred} updated with VAE layer and attention layer, and BERT\cite{zeng2020double} updated with ...

\section{Evaluation}
% (4) describe how you evaluated the approaches (e.g., using which metrics); 
% here we should find out the popular evaluations in previous papers, such as precision, accuracy or F1 score. 
Previous papers all use F1 score to evaluate the methods work on relation extraction (RE).

\section{Results}
% (5) discuss the results of your evaluation, comparing the approaches and highlighting their strengths and weaknesses.

\bibliography{references}
\bibliographystyle{unsrt}

\end{document}